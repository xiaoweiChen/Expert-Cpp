不同编程语言的执行模型也不同,最常见的便是解释语言和编译语言。编译器将源代码翻译成机器代码,计算机可以在没有中间系统支持的环境下运行。另外,解释性语言代码需要支持系统、解释器和虚拟环境才能工作。 \par
C++是编译语言,所以会比解释型程序运行得更快。但C++程序需要针对每个平台进行编译,但解释型程序可以跨平台运行。 \par
我们将讨论程序构建的细节,从源代码阶段开始——由编译器完成——到可执行文件(编译器的输出)结束。还会去了解,为什么为一个平台构建的程序不能在另一个平台上运行。 \par
本章将讨论以下主题: \par

\begin{itemize}
	\item 介绍C++20。
	\item C++预处理的细节。
	\item 源代码的(底层)编译。
	\item 了解连接器及其功能。
	\item 加载和运行可执行文件的过程。
\end{itemize}



