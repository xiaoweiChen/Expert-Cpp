游戏开发是软件工程中最有趣的话题之一。C++因其高效而广泛应用于游戏开发中,但由于言没有GUI组件,所以在后端使用。本章中,我们将学习如何在后端设计策略游戏。我们将把前面章节中学到的所有东西都结合起来,包括设计模式和多线程。 \par
我们要设计的游戏是一款名为《读者与扰乱者》的策略游戏。玩家创造了能够建造图书馆和其他建筑的单位,也就是所谓的“读者”,以及守卫这些建筑不受敌人攻击的士兵。 \par
本章中,我们将了解以下内容: \par

\begin{itemize}
	\item 游戏制作入门
	\item 深入研究游戏设计过程
	\item 使用设计模式
	\item 设计游戏循环
\end{itemize}

\noindent\textbf{}\ \par
\textbf{编译器要求} \ \par
g++编译器需要添加编译选项 \texttt{-std=c++2a} 来编译本章的代码。可以从这里获取本章的源码文件:https:/​/github.​com/PacktPublishing/Expert-CPP \par

\noindent\textbf{}\ \par
\textbf{游戏制作入门} \ \par

\noindent\textbf{}\ \par
\textbf{了解游戏} \ \par

\noindent\textbf{}\ \par
\textbf{战略游戏组件} \ \par

\noindent\textbf{}\ \par
\textbf{组件之间的相互作用} \ \par

\noindent\textbf{}\ \par
\textbf{设计游戏} \ \par

\noindent\textbf{}\ \par
\textbf{设计角色} \ \par

\noindent\textbf{}\ \par
\textbf{设计建筑} \ \par

\noindent\textbf{}\ \par
\textbf{设计游戏控制器} \ \par

\noindent\textbf{}\ \par
\textbf{游戏事件循环} \ \par

\noindent\textbf{}\ \par
\textbf{使用设计模式} \ \par

\noindent\textbf{}\ \par
\textbf{命令模式} \ \par

\noindent\textbf{}\ \par
\textbf{观察者模式} \ \par

\noindent\textbf{}\ \par
\textbf{享元模式} \ \par

\noindent\textbf{}\ \par
\textbf{原型模式} \ \par

\noindent\textbf{}\ \par
\textbf{设计游戏循环} \ \par

\noindent\textbf{}\ \par
\textbf{总结} \ \par

\noindent\textbf{}\ \par
\textbf{问题} \ \par
\begin{enumerate}
	\item 重写一个私有虚拟函数的目的是什么?
	\item 描述命令设计模式。
	\item 享元模式如何节省内存使用?
	\item 观察者模式和中介模式之间有什么区别?
	\item 为什么我们要将游戏循环设计成一个无限循环?
\end{enumerate}

\noindent\textbf{}\ \par
\textbf{扩展阅读} \ \par
\begin{itemize}
	\item Game Development Patterns and Best Practices: Better games, less hassle by John P.Doran, Matt Casanova:  https:/​/​www.​amazon.​com/​Game-​Development-​Patterns-	Best-​Practices/​dp/​1787127834/​ .
\end{itemize}

\newpage









