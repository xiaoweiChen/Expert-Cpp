\begin{enumerate}
	\item 如果两个操作的开始时间和结束时间在任意点交叉,那么它们将并发运行。
	\item 并行性意味着任务同时运行,而并发性并不强制任务同时运行。
	\item 过程是程序的映像。它是程序指令和装入计算机内存的数据的组合。
	\item 线程是进程范围内的一段代码,可以由操作系统调度器调度,而进程是正在运行的程序的镜像。
	\item 参考本章中的例子。
	\item 通过使用双重检查锁定。
	\item 参考GitHub中这一章的源代码。
	\item C++20引入协程作为对经典异步函数的补充。协程将代码的后台执行移动到下一个级别,允许一个函数在必要时被暂停和恢复。co\underline{ }await是一个告诉代码等待异步执行代码的构造。这意味着函数可以在此时挂起,并在结果就绪时继续执行。
\end{enumerate}












