\begin{enumerate}
	\item 当向vector对象插入新元素时,它将被放置在vector对象中已经分配的空闲槽位上。如果vector的大小与容量相等,则意味着vector没有存放新元素的空闲槽位。在这些(罕见的)情况下,vector会自动调整自身的大小,这涉及到分配新的内存空间并将现有元素复制到新的更大的空间。
	\item 当在链表的前面插入元素时,我们只创建新元素并更新链表指针,以有效地将新元素放入链表中。在vector的前端插入新元素需要将所有vector元素向右移动,从而为该元素腾出一个槽位。
	\item 选择排序搜索最大(或最小)元素,并用该最大(或最小)元素替换当前元素。插入排序将集合分成两个部分,遍历未排序的部分,并将其每个元素放置在已排序部分的适当槽中。
	\item 参考GitHub中这一章的源代码。
\end{enumerate}












