\begin{enumerate}
	\item TDD代表测试驱动开发,其目标是在项目的实际实现之前编写测试。这有助于更清楚地定义项目需求,并预先避免代码中的大多数错误。
	\item 交互图描绘了对象之间交流的准确过程。这允许开发人员对任何给定时刻的实际程序执行有一个高级视图。
	\item 在聚合的情况下,可以在没有聚合的情况下实例化包含一个或多个其他类实例的类。
	\item 简单地说,Liskov替换原则确保任何函数接受某种类型T的对象作为参数,如果K扩展了T,也将接受类型K的对象。
	\item 开闭原则表示类应该对扩展开放,对修改关闭。在上述例子中,Animal是可扩展的,因此它与从Animal继承monkey类的原则并不矛盾。
	\item 参考GitHub中这一章的源代码。
\end{enumerate}












